
% Default to the notebook output style

    


% Inherit from the specified cell style.




    
\documentclass[11pt]{article}

    
    
    \usepackage[T1]{fontenc}
    % Nicer default font (+ math font) than Computer Modern for most use cases
    \usepackage{mathpazo}

    % Basic figure setup, for now with no caption control since it's done
    % automatically by Pandoc (which extracts ![](path) syntax from Markdown).
    \usepackage{graphicx}
    % We will generate all images so they have a width \maxwidth. This means
    % that they will get their normal width if they fit onto the page, but
    % are scaled down if they would overflow the margins.
    \makeatletter
    \def\maxwidth{\ifdim\Gin@nat@width>\linewidth\linewidth
    \else\Gin@nat@width\fi}
    \makeatother
    \let\Oldincludegraphics\includegraphics
    % Set max figure width to be 80% of text width, for now hardcoded.
    \renewcommand{\includegraphics}[1]{\Oldincludegraphics[width=.8\maxwidth]{#1}}
    % Ensure that by default, figures have no caption (until we provide a
    % proper Figure object with a Caption API and a way to capture that
    % in the conversion process - todo).
    \usepackage{caption}
    \DeclareCaptionLabelFormat{nolabel}{}
    \captionsetup{labelformat=nolabel}

    \usepackage{adjustbox} % Used to constrain images to a maximum size 
    \usepackage{xcolor} % Allow colors to be defined
    \usepackage{enumerate} % Needed for markdown enumerations to work
    \usepackage{geometry} % Used to adjust the document margins
    \usepackage{amsmath} % Equations
    \usepackage{amssymb} % Equations
    \usepackage{textcomp} % defines textquotesingle
    % Hack from http://tex.stackexchange.com/a/47451/13684:
    \AtBeginDocument{%
        \def\PYZsq{\textquotesingle}% Upright quotes in Pygmentized code
    }
    \usepackage{upquote} % Upright quotes for verbatim code
    \usepackage{eurosym} % defines \euro
    \usepackage[mathletters]{ucs} % Extended unicode (utf-8) support
    \usepackage[utf8x]{inputenc} % Allow utf-8 characters in the tex document
    \usepackage{fancyvrb} % verbatim replacement that allows latex
    \usepackage{grffile} % extends the file name processing of package graphics 
                         % to support a larger range 
    % The hyperref package gives us a pdf with properly built
    % internal navigation ('pdf bookmarks' for the table of contents,
    % internal cross-reference links, web links for URLs, etc.)
    \usepackage{hyperref}
    \usepackage{longtable} % longtable support required by pandoc >1.10
    \usepackage{booktabs}  % table support for pandoc > 1.12.2
    \usepackage[inline]{enumitem} % IRkernel/repr support (it uses the enumerate* environment)
    \usepackage[normalem]{ulem} % ulem is needed to support strikethroughs (\sout)
                                % normalem makes italics be italics, not underlines
    

    
    
    % Colors for the hyperref package
    \definecolor{urlcolor}{rgb}{0,.145,.698}
    \definecolor{linkcolor}{rgb}{.71,0.21,0.01}
    \definecolor{citecolor}{rgb}{.12,.54,.11}

    % ANSI colors
    \definecolor{ansi-black}{HTML}{3E424D}
    \definecolor{ansi-black-intense}{HTML}{282C36}
    \definecolor{ansi-red}{HTML}{E75C58}
    \definecolor{ansi-red-intense}{HTML}{B22B31}
    \definecolor{ansi-green}{HTML}{00A250}
    \definecolor{ansi-green-intense}{HTML}{007427}
    \definecolor{ansi-yellow}{HTML}{DDB62B}
    \definecolor{ansi-yellow-intense}{HTML}{B27D12}
    \definecolor{ansi-blue}{HTML}{208FFB}
    \definecolor{ansi-blue-intense}{HTML}{0065CA}
    \definecolor{ansi-magenta}{HTML}{D160C4}
    \definecolor{ansi-magenta-intense}{HTML}{A03196}
    \definecolor{ansi-cyan}{HTML}{60C6C8}
    \definecolor{ansi-cyan-intense}{HTML}{258F8F}
    \definecolor{ansi-white}{HTML}{C5C1B4}
    \definecolor{ansi-white-intense}{HTML}{A1A6B2}

    % commands and environments needed by pandoc snippets
    % extracted from the output of `pandoc -s`
    \providecommand{\tightlist}{%
      \setlength{\itemsep}{0pt}\setlength{\parskip}{0pt}}
    \DefineVerbatimEnvironment{Highlighting}{Verbatim}{commandchars=\\\{\}}
    % Add ',fontsize=\small' for more characters per line
    \newenvironment{Shaded}{}{}
    \newcommand{\KeywordTok}[1]{\textcolor[rgb]{0.00,0.44,0.13}{\textbf{{#1}}}}
    \newcommand{\DataTypeTok}[1]{\textcolor[rgb]{0.56,0.13,0.00}{{#1}}}
    \newcommand{\DecValTok}[1]{\textcolor[rgb]{0.25,0.63,0.44}{{#1}}}
    \newcommand{\BaseNTok}[1]{\textcolor[rgb]{0.25,0.63,0.44}{{#1}}}
    \newcommand{\FloatTok}[1]{\textcolor[rgb]{0.25,0.63,0.44}{{#1}}}
    \newcommand{\CharTok}[1]{\textcolor[rgb]{0.25,0.44,0.63}{{#1}}}
    \newcommand{\StringTok}[1]{\textcolor[rgb]{0.25,0.44,0.63}{{#1}}}
    \newcommand{\CommentTok}[1]{\textcolor[rgb]{0.38,0.63,0.69}{\textit{{#1}}}}
    \newcommand{\OtherTok}[1]{\textcolor[rgb]{0.00,0.44,0.13}{{#1}}}
    \newcommand{\AlertTok}[1]{\textcolor[rgb]{1.00,0.00,0.00}{\textbf{{#1}}}}
    \newcommand{\FunctionTok}[1]{\textcolor[rgb]{0.02,0.16,0.49}{{#1}}}
    \newcommand{\RegionMarkerTok}[1]{{#1}}
    \newcommand{\ErrorTok}[1]{\textcolor[rgb]{1.00,0.00,0.00}{\textbf{{#1}}}}
    \newcommand{\NormalTok}[1]{{#1}}
    
    % Additional commands for more recent versions of Pandoc
    \newcommand{\ConstantTok}[1]{\textcolor[rgb]{0.53,0.00,0.00}{{#1}}}
    \newcommand{\SpecialCharTok}[1]{\textcolor[rgb]{0.25,0.44,0.63}{{#1}}}
    \newcommand{\VerbatimStringTok}[1]{\textcolor[rgb]{0.25,0.44,0.63}{{#1}}}
    \newcommand{\SpecialStringTok}[1]{\textcolor[rgb]{0.73,0.40,0.53}{{#1}}}
    \newcommand{\ImportTok}[1]{{#1}}
    \newcommand{\DocumentationTok}[1]{\textcolor[rgb]{0.73,0.13,0.13}{\textit{{#1}}}}
    \newcommand{\AnnotationTok}[1]{\textcolor[rgb]{0.38,0.63,0.69}{\textbf{\textit{{#1}}}}}
    \newcommand{\CommentVarTok}[1]{\textcolor[rgb]{0.38,0.63,0.69}{\textbf{\textit{{#1}}}}}
    \newcommand{\VariableTok}[1]{\textcolor[rgb]{0.10,0.09,0.49}{{#1}}}
    \newcommand{\ControlFlowTok}[1]{\textcolor[rgb]{0.00,0.44,0.13}{\textbf{{#1}}}}
    \newcommand{\OperatorTok}[1]{\textcolor[rgb]{0.40,0.40,0.40}{{#1}}}
    \newcommand{\BuiltInTok}[1]{{#1}}
    \newcommand{\ExtensionTok}[1]{{#1}}
    \newcommand{\PreprocessorTok}[1]{\textcolor[rgb]{0.74,0.48,0.00}{{#1}}}
    \newcommand{\AttributeTok}[1]{\textcolor[rgb]{0.49,0.56,0.16}{{#1}}}
    \newcommand{\InformationTok}[1]{\textcolor[rgb]{0.38,0.63,0.69}{\textbf{\textit{{#1}}}}}
    \newcommand{\WarningTok}[1]{\textcolor[rgb]{0.38,0.63,0.69}{\textbf{\textit{{#1}}}}}
    
    
    % Define a nice break command that doesn't care if a line doesn't already
    % exist.
    \def\br{\hspace*{\fill} \\* }
    % Math Jax compatability definitions
    \def\gt{>}
    \def\lt{<}
    % Document parameters
    \title{01-Support Vector Machines with Python}
    
    
    

    % Pygments definitions
    
\makeatletter
\def\PY@reset{\let\PY@it=\relax \let\PY@bf=\relax%
    \let\PY@ul=\relax \let\PY@tc=\relax%
    \let\PY@bc=\relax \let\PY@ff=\relax}
\def\PY@tok#1{\csname PY@tok@#1\endcsname}
\def\PY@toks#1+{\ifx\relax#1\empty\else%
    \PY@tok{#1}\expandafter\PY@toks\fi}
\def\PY@do#1{\PY@bc{\PY@tc{\PY@ul{%
    \PY@it{\PY@bf{\PY@ff{#1}}}}}}}
\def\PY#1#2{\PY@reset\PY@toks#1+\relax+\PY@do{#2}}

\expandafter\def\csname PY@tok@w\endcsname{\def\PY@tc##1{\textcolor[rgb]{0.73,0.73,0.73}{##1}}}
\expandafter\def\csname PY@tok@c\endcsname{\let\PY@it=\textit\def\PY@tc##1{\textcolor[rgb]{0.25,0.50,0.50}{##1}}}
\expandafter\def\csname PY@tok@cp\endcsname{\def\PY@tc##1{\textcolor[rgb]{0.74,0.48,0.00}{##1}}}
\expandafter\def\csname PY@tok@k\endcsname{\let\PY@bf=\textbf\def\PY@tc##1{\textcolor[rgb]{0.00,0.50,0.00}{##1}}}
\expandafter\def\csname PY@tok@kp\endcsname{\def\PY@tc##1{\textcolor[rgb]{0.00,0.50,0.00}{##1}}}
\expandafter\def\csname PY@tok@kt\endcsname{\def\PY@tc##1{\textcolor[rgb]{0.69,0.00,0.25}{##1}}}
\expandafter\def\csname PY@tok@o\endcsname{\def\PY@tc##1{\textcolor[rgb]{0.40,0.40,0.40}{##1}}}
\expandafter\def\csname PY@tok@ow\endcsname{\let\PY@bf=\textbf\def\PY@tc##1{\textcolor[rgb]{0.67,0.13,1.00}{##1}}}
\expandafter\def\csname PY@tok@nb\endcsname{\def\PY@tc##1{\textcolor[rgb]{0.00,0.50,0.00}{##1}}}
\expandafter\def\csname PY@tok@nf\endcsname{\def\PY@tc##1{\textcolor[rgb]{0.00,0.00,1.00}{##1}}}
\expandafter\def\csname PY@tok@nc\endcsname{\let\PY@bf=\textbf\def\PY@tc##1{\textcolor[rgb]{0.00,0.00,1.00}{##1}}}
\expandafter\def\csname PY@tok@nn\endcsname{\let\PY@bf=\textbf\def\PY@tc##1{\textcolor[rgb]{0.00,0.00,1.00}{##1}}}
\expandafter\def\csname PY@tok@ne\endcsname{\let\PY@bf=\textbf\def\PY@tc##1{\textcolor[rgb]{0.82,0.25,0.23}{##1}}}
\expandafter\def\csname PY@tok@nv\endcsname{\def\PY@tc##1{\textcolor[rgb]{0.10,0.09,0.49}{##1}}}
\expandafter\def\csname PY@tok@no\endcsname{\def\PY@tc##1{\textcolor[rgb]{0.53,0.00,0.00}{##1}}}
\expandafter\def\csname PY@tok@nl\endcsname{\def\PY@tc##1{\textcolor[rgb]{0.63,0.63,0.00}{##1}}}
\expandafter\def\csname PY@tok@ni\endcsname{\let\PY@bf=\textbf\def\PY@tc##1{\textcolor[rgb]{0.60,0.60,0.60}{##1}}}
\expandafter\def\csname PY@tok@na\endcsname{\def\PY@tc##1{\textcolor[rgb]{0.49,0.56,0.16}{##1}}}
\expandafter\def\csname PY@tok@nt\endcsname{\let\PY@bf=\textbf\def\PY@tc##1{\textcolor[rgb]{0.00,0.50,0.00}{##1}}}
\expandafter\def\csname PY@tok@nd\endcsname{\def\PY@tc##1{\textcolor[rgb]{0.67,0.13,1.00}{##1}}}
\expandafter\def\csname PY@tok@s\endcsname{\def\PY@tc##1{\textcolor[rgb]{0.73,0.13,0.13}{##1}}}
\expandafter\def\csname PY@tok@sd\endcsname{\let\PY@it=\textit\def\PY@tc##1{\textcolor[rgb]{0.73,0.13,0.13}{##1}}}
\expandafter\def\csname PY@tok@si\endcsname{\let\PY@bf=\textbf\def\PY@tc##1{\textcolor[rgb]{0.73,0.40,0.53}{##1}}}
\expandafter\def\csname PY@tok@se\endcsname{\let\PY@bf=\textbf\def\PY@tc##1{\textcolor[rgb]{0.73,0.40,0.13}{##1}}}
\expandafter\def\csname PY@tok@sr\endcsname{\def\PY@tc##1{\textcolor[rgb]{0.73,0.40,0.53}{##1}}}
\expandafter\def\csname PY@tok@ss\endcsname{\def\PY@tc##1{\textcolor[rgb]{0.10,0.09,0.49}{##1}}}
\expandafter\def\csname PY@tok@sx\endcsname{\def\PY@tc##1{\textcolor[rgb]{0.00,0.50,0.00}{##1}}}
\expandafter\def\csname PY@tok@m\endcsname{\def\PY@tc##1{\textcolor[rgb]{0.40,0.40,0.40}{##1}}}
\expandafter\def\csname PY@tok@gh\endcsname{\let\PY@bf=\textbf\def\PY@tc##1{\textcolor[rgb]{0.00,0.00,0.50}{##1}}}
\expandafter\def\csname PY@tok@gu\endcsname{\let\PY@bf=\textbf\def\PY@tc##1{\textcolor[rgb]{0.50,0.00,0.50}{##1}}}
\expandafter\def\csname PY@tok@gd\endcsname{\def\PY@tc##1{\textcolor[rgb]{0.63,0.00,0.00}{##1}}}
\expandafter\def\csname PY@tok@gi\endcsname{\def\PY@tc##1{\textcolor[rgb]{0.00,0.63,0.00}{##1}}}
\expandafter\def\csname PY@tok@gr\endcsname{\def\PY@tc##1{\textcolor[rgb]{1.00,0.00,0.00}{##1}}}
\expandafter\def\csname PY@tok@ge\endcsname{\let\PY@it=\textit}
\expandafter\def\csname PY@tok@gs\endcsname{\let\PY@bf=\textbf}
\expandafter\def\csname PY@tok@gp\endcsname{\let\PY@bf=\textbf\def\PY@tc##1{\textcolor[rgb]{0.00,0.00,0.50}{##1}}}
\expandafter\def\csname PY@tok@go\endcsname{\def\PY@tc##1{\textcolor[rgb]{0.53,0.53,0.53}{##1}}}
\expandafter\def\csname PY@tok@gt\endcsname{\def\PY@tc##1{\textcolor[rgb]{0.00,0.27,0.87}{##1}}}
\expandafter\def\csname PY@tok@err\endcsname{\def\PY@bc##1{\setlength{\fboxsep}{0pt}\fcolorbox[rgb]{1.00,0.00,0.00}{1,1,1}{\strut ##1}}}
\expandafter\def\csname PY@tok@kc\endcsname{\let\PY@bf=\textbf\def\PY@tc##1{\textcolor[rgb]{0.00,0.50,0.00}{##1}}}
\expandafter\def\csname PY@tok@kd\endcsname{\let\PY@bf=\textbf\def\PY@tc##1{\textcolor[rgb]{0.00,0.50,0.00}{##1}}}
\expandafter\def\csname PY@tok@kn\endcsname{\let\PY@bf=\textbf\def\PY@tc##1{\textcolor[rgb]{0.00,0.50,0.00}{##1}}}
\expandafter\def\csname PY@tok@kr\endcsname{\let\PY@bf=\textbf\def\PY@tc##1{\textcolor[rgb]{0.00,0.50,0.00}{##1}}}
\expandafter\def\csname PY@tok@bp\endcsname{\def\PY@tc##1{\textcolor[rgb]{0.00,0.50,0.00}{##1}}}
\expandafter\def\csname PY@tok@fm\endcsname{\def\PY@tc##1{\textcolor[rgb]{0.00,0.00,1.00}{##1}}}
\expandafter\def\csname PY@tok@vc\endcsname{\def\PY@tc##1{\textcolor[rgb]{0.10,0.09,0.49}{##1}}}
\expandafter\def\csname PY@tok@vg\endcsname{\def\PY@tc##1{\textcolor[rgb]{0.10,0.09,0.49}{##1}}}
\expandafter\def\csname PY@tok@vi\endcsname{\def\PY@tc##1{\textcolor[rgb]{0.10,0.09,0.49}{##1}}}
\expandafter\def\csname PY@tok@vm\endcsname{\def\PY@tc##1{\textcolor[rgb]{0.10,0.09,0.49}{##1}}}
\expandafter\def\csname PY@tok@sa\endcsname{\def\PY@tc##1{\textcolor[rgb]{0.73,0.13,0.13}{##1}}}
\expandafter\def\csname PY@tok@sb\endcsname{\def\PY@tc##1{\textcolor[rgb]{0.73,0.13,0.13}{##1}}}
\expandafter\def\csname PY@tok@sc\endcsname{\def\PY@tc##1{\textcolor[rgb]{0.73,0.13,0.13}{##1}}}
\expandafter\def\csname PY@tok@dl\endcsname{\def\PY@tc##1{\textcolor[rgb]{0.73,0.13,0.13}{##1}}}
\expandafter\def\csname PY@tok@s2\endcsname{\def\PY@tc##1{\textcolor[rgb]{0.73,0.13,0.13}{##1}}}
\expandafter\def\csname PY@tok@sh\endcsname{\def\PY@tc##1{\textcolor[rgb]{0.73,0.13,0.13}{##1}}}
\expandafter\def\csname PY@tok@s1\endcsname{\def\PY@tc##1{\textcolor[rgb]{0.73,0.13,0.13}{##1}}}
\expandafter\def\csname PY@tok@mb\endcsname{\def\PY@tc##1{\textcolor[rgb]{0.40,0.40,0.40}{##1}}}
\expandafter\def\csname PY@tok@mf\endcsname{\def\PY@tc##1{\textcolor[rgb]{0.40,0.40,0.40}{##1}}}
\expandafter\def\csname PY@tok@mh\endcsname{\def\PY@tc##1{\textcolor[rgb]{0.40,0.40,0.40}{##1}}}
\expandafter\def\csname PY@tok@mi\endcsname{\def\PY@tc##1{\textcolor[rgb]{0.40,0.40,0.40}{##1}}}
\expandafter\def\csname PY@tok@il\endcsname{\def\PY@tc##1{\textcolor[rgb]{0.40,0.40,0.40}{##1}}}
\expandafter\def\csname PY@tok@mo\endcsname{\def\PY@tc##1{\textcolor[rgb]{0.40,0.40,0.40}{##1}}}
\expandafter\def\csname PY@tok@ch\endcsname{\let\PY@it=\textit\def\PY@tc##1{\textcolor[rgb]{0.25,0.50,0.50}{##1}}}
\expandafter\def\csname PY@tok@cm\endcsname{\let\PY@it=\textit\def\PY@tc##1{\textcolor[rgb]{0.25,0.50,0.50}{##1}}}
\expandafter\def\csname PY@tok@cpf\endcsname{\let\PY@it=\textit\def\PY@tc##1{\textcolor[rgb]{0.25,0.50,0.50}{##1}}}
\expandafter\def\csname PY@tok@c1\endcsname{\let\PY@it=\textit\def\PY@tc##1{\textcolor[rgb]{0.25,0.50,0.50}{##1}}}
\expandafter\def\csname PY@tok@cs\endcsname{\let\PY@it=\textit\def\PY@tc##1{\textcolor[rgb]{0.25,0.50,0.50}{##1}}}

\def\PYZbs{\char`\\}
\def\PYZus{\char`\_}
\def\PYZob{\char`\{}
\def\PYZcb{\char`\}}
\def\PYZca{\char`\^}
\def\PYZam{\char`\&}
\def\PYZlt{\char`\<}
\def\PYZgt{\char`\>}
\def\PYZsh{\char`\#}
\def\PYZpc{\char`\%}
\def\PYZdl{\char`\$}
\def\PYZhy{\char`\-}
\def\PYZsq{\char`\'}
\def\PYZdq{\char`\"}
\def\PYZti{\char`\~}
% for compatibility with earlier versions
\def\PYZat{@}
\def\PYZlb{[}
\def\PYZrb{]}
\makeatother


    % Exact colors from NB
    \definecolor{incolor}{rgb}{0.0, 0.0, 0.5}
    \definecolor{outcolor}{rgb}{0.545, 0.0, 0.0}



    
    % Prevent overflowing lines due to hard-to-break entities
    \sloppy 
    % Setup hyperref package
    \hypersetup{
      breaklinks=true,  % so long urls are correctly broken across lines
      colorlinks=true,
      urlcolor=urlcolor,
      linkcolor=linkcolor,
      citecolor=citecolor,
      }
    % Slightly bigger margins than the latex defaults
    
    \geometry{verbose,tmargin=1in,bmargin=1in,lmargin=1in,rmargin=1in}
    
    

    \begin{document}
    
    
    \maketitle
    
    

    
    \section{Support Vector Machines with
Python}\label{support-vector-machines-with-python}

Welcome to the Support Vector Machines with Python Lecture Notebook!

\subsection{Import Libraries}\label{import-libraries}

    \begin{Verbatim}[commandchars=\\\{\}]
{\color{incolor}In [{\color{incolor}1}]:} \PY{k+kn}{import} \PY{n+nn}{pandas} \PY{k}{as} \PY{n+nn}{pd}
        \PY{k+kn}{import} \PY{n+nn}{numpy} \PY{k}{as} \PY{n+nn}{np}
        \PY{k+kn}{import} \PY{n+nn}{matplotlib}\PY{n+nn}{.}\PY{n+nn}{pyplot} \PY{k}{as} \PY{n+nn}{plt}
        \PY{k+kn}{import} \PY{n+nn}{seaborn} \PY{k}{as} \PY{n+nn}{sns}
        \PY{o}{\PYZpc{}}\PY{k}{matplotlib} inline
\end{Verbatim}


    \subsection{Get the Data}\label{get-the-data}

We'll use the built in breast cancer dataset from Scikit Learn. We can
get with the load function:

    \begin{Verbatim}[commandchars=\\\{\}]
{\color{incolor}In [{\color{incolor}2}]:} \PY{k+kn}{from} \PY{n+nn}{sklearn}\PY{n+nn}{.}\PY{n+nn}{datasets} \PY{k}{import} \PY{n}{load\PYZus{}breast\PYZus{}cancer}
\end{Verbatim}


    \begin{Verbatim}[commandchars=\\\{\}]
{\color{incolor}In [{\color{incolor}3}]:} \PY{n}{cancer} \PY{o}{=} \PY{n}{load\PYZus{}breast\PYZus{}cancer}\PY{p}{(}\PY{p}{)}
\end{Verbatim}


    The data set is presented in a dictionary form:

    \begin{Verbatim}[commandchars=\\\{\}]
{\color{incolor}In [{\color{incolor}4}]:} \PY{n}{cancer}\PY{o}{.}\PY{n}{keys}\PY{p}{(}\PY{p}{)}
\end{Verbatim}


\begin{Verbatim}[commandchars=\\\{\}]
{\color{outcolor}Out[{\color{outcolor}4}]:} dict\_keys(['data', 'target', 'target\_names', 'DESCR', 'feature\_names', 'filename'])
\end{Verbatim}
            
    We can grab information and arrays out of this dictionary to set up our
data frame and understanding of the features:

    \begin{Verbatim}[commandchars=\\\{\}]
{\color{incolor}In [{\color{incolor}5}]:} \PY{n+nb}{print}\PY{p}{(}\PY{n}{cancer}\PY{p}{[}\PY{l+s+s1}{\PYZsq{}}\PY{l+s+s1}{DESCR}\PY{l+s+s1}{\PYZsq{}}\PY{p}{]}\PY{p}{)}
\end{Verbatim}


    \begin{Verbatim}[commandchars=\\\{\}]
.. \_breast\_cancer\_dataset:

Breast cancer wisconsin (diagnostic) dataset
--------------------------------------------

**Data Set Characteristics:**

    :Number of Instances: 569

    :Number of Attributes: 30 numeric, predictive attributes and the class

    :Attribute Information:
        - radius (mean of distances from center to points on the perimeter)
        - texture (standard deviation of gray-scale values)
        - perimeter
        - area
        - smoothness (local variation in radius lengths)
        - compactness (perimeter\^{}2 / area - 1.0)
        - concavity (severity of concave portions of the contour)
        - concave points (number of concave portions of the contour)
        - symmetry 
        - fractal dimension ("coastline approximation" - 1)

        The mean, standard error, and "worst" or largest (mean of the three
        largest values) of these features were computed for each image,
        resulting in 30 features.  For instance, field 3 is Mean Radius, field
        13 is Radius SE, field 23 is Worst Radius.

        - class:
                - WDBC-Malignant
                - WDBC-Benign

    :Summary Statistics:

    ===================================== ====== ======
                                           Min    Max
    ===================================== ====== ======
    radius (mean):                        6.981  28.11
    texture (mean):                       9.71   39.28
    perimeter (mean):                     43.79  188.5
    area (mean):                          143.5  2501.0
    smoothness (mean):                    0.053  0.163
    compactness (mean):                   0.019  0.345
    concavity (mean):                     0.0    0.427
    concave points (mean):                0.0    0.201
    symmetry (mean):                      0.106  0.304
    fractal dimension (mean):             0.05   0.097
    radius (standard error):              0.112  2.873
    texture (standard error):             0.36   4.885
    perimeter (standard error):           0.757  21.98
    area (standard error):                6.802  542.2
    smoothness (standard error):          0.002  0.031
    compactness (standard error):         0.002  0.135
    concavity (standard error):           0.0    0.396
    concave points (standard error):      0.0    0.053
    symmetry (standard error):            0.008  0.079
    fractal dimension (standard error):   0.001  0.03
    radius (worst):                       7.93   36.04
    texture (worst):                      12.02  49.54
    perimeter (worst):                    50.41  251.2
    area (worst):                         185.2  4254.0
    smoothness (worst):                   0.071  0.223
    compactness (worst):                  0.027  1.058
    concavity (worst):                    0.0    1.252
    concave points (worst):               0.0    0.291
    symmetry (worst):                     0.156  0.664
    fractal dimension (worst):            0.055  0.208
    ===================================== ====== ======

    :Missing Attribute Values: None

    :Class Distribution: 212 - Malignant, 357 - Benign

    :Creator:  Dr. William H. Wolberg, W. Nick Street, Olvi L. Mangasarian

    :Donor: Nick Street

    :Date: November, 1995

This is a copy of UCI ML Breast Cancer Wisconsin (Diagnostic) datasets.
https://goo.gl/U2Uwz2

Features are computed from a digitized image of a fine needle
aspirate (FNA) of a breast mass.  They describe
characteristics of the cell nuclei present in the image.

Separating plane described above was obtained using
Multisurface Method-Tree (MSM-T) [K. P. Bennett, "Decision Tree
Construction Via Linear Programming." Proceedings of the 4th
Midwest Artificial Intelligence and Cognitive Science Society,
pp. 97-101, 1992], a classification method which uses linear
programming to construct a decision tree.  Relevant features
were selected using an exhaustive search in the space of 1-4
features and 1-3 separating planes.

The actual linear program used to obtain the separating plane
in the 3-dimensional space is that described in:
[K. P. Bennett and O. L. Mangasarian: "Robust Linear
Programming Discrimination of Two Linearly Inseparable Sets",
Optimization Methods and Software 1, 1992, 23-34].

This database is also available through the UW CS ftp server:

ftp ftp.cs.wisc.edu
cd math-prog/cpo-dataset/machine-learn/WDBC/

.. topic:: References

   - W.N. Street, W.H. Wolberg and O.L. Mangasarian. Nuclear feature extraction 
     for breast tumor diagnosis. IS\&T/SPIE 1993 International Symposium on 
     Electronic Imaging: Science and Technology, volume 1905, pages 861-870,
     San Jose, CA, 1993.
   - O.L. Mangasarian, W.N. Street and W.H. Wolberg. Breast cancer diagnosis and 
     prognosis via linear programming. Operations Research, 43(4), pages 570-577, 
     July-August 1995.
   - W.H. Wolberg, W.N. Street, and O.L. Mangasarian. Machine learning techniques
     to diagnose breast cancer from fine-needle aspirates. Cancer Letters 77 (1994) 
     163-171.

    \end{Verbatim}

    \begin{Verbatim}[commandchars=\\\{\}]
{\color{incolor}In [{\color{incolor}6}]:} \PY{n}{cancer}\PY{p}{[}\PY{l+s+s1}{\PYZsq{}}\PY{l+s+s1}{feature\PYZus{}names}\PY{l+s+s1}{\PYZsq{}}\PY{p}{]}
\end{Verbatim}


\begin{Verbatim}[commandchars=\\\{\}]
{\color{outcolor}Out[{\color{outcolor}6}]:} array(['mean radius', 'mean texture', 'mean perimeter', 'mean area',
               'mean smoothness', 'mean compactness', 'mean concavity',
               'mean concave points', 'mean symmetry', 'mean fractal dimension',
               'radius error', 'texture error', 'perimeter error', 'area error',
               'smoothness error', 'compactness error', 'concavity error',
               'concave points error', 'symmetry error',
               'fractal dimension error', 'worst radius', 'worst texture',
               'worst perimeter', 'worst area', 'worst smoothness',
               'worst compactness', 'worst concavity', 'worst concave points',
               'worst symmetry', 'worst fractal dimension'], dtype='<U23')
\end{Verbatim}
            
    \subsection{Set up DataFrame}\label{set-up-dataframe}

    \begin{Verbatim}[commandchars=\\\{\}]
{\color{incolor}In [{\color{incolor}7}]:} \PY{n}{df\PYZus{}feat} \PY{o}{=} \PY{n}{pd}\PY{o}{.}\PY{n}{DataFrame}\PY{p}{(}\PY{n}{cancer}\PY{p}{[}\PY{l+s+s1}{\PYZsq{}}\PY{l+s+s1}{data}\PY{l+s+s1}{\PYZsq{}}\PY{p}{]}\PY{p}{,}\PY{n}{columns}\PY{o}{=}\PY{n}{cancer}\PY{p}{[}\PY{l+s+s1}{\PYZsq{}}\PY{l+s+s1}{feature\PYZus{}names}\PY{l+s+s1}{\PYZsq{}}\PY{p}{]}\PY{p}{)}
        \PY{n}{df\PYZus{}feat}\PY{o}{.}\PY{n}{info}\PY{p}{(}\PY{p}{)}
\end{Verbatim}


    \begin{Verbatim}[commandchars=\\\{\}]
<class 'pandas.core.frame.DataFrame'>
RangeIndex: 569 entries, 0 to 568
Data columns (total 30 columns):
mean radius                569 non-null float64
mean texture               569 non-null float64
mean perimeter             569 non-null float64
mean area                  569 non-null float64
mean smoothness            569 non-null float64
mean compactness           569 non-null float64
mean concavity             569 non-null float64
mean concave points        569 non-null float64
mean symmetry              569 non-null float64
mean fractal dimension     569 non-null float64
radius error               569 non-null float64
texture error              569 non-null float64
perimeter error            569 non-null float64
area error                 569 non-null float64
smoothness error           569 non-null float64
compactness error          569 non-null float64
concavity error            569 non-null float64
concave points error       569 non-null float64
symmetry error             569 non-null float64
fractal dimension error    569 non-null float64
worst radius               569 non-null float64
worst texture              569 non-null float64
worst perimeter            569 non-null float64
worst area                 569 non-null float64
worst smoothness           569 non-null float64
worst compactness          569 non-null float64
worst concavity            569 non-null float64
worst concave points       569 non-null float64
worst symmetry             569 non-null float64
worst fractal dimension    569 non-null float64
dtypes: float64(30)
memory usage: 133.4 KB

    \end{Verbatim}

    \begin{Verbatim}[commandchars=\\\{\}]
{\color{incolor}In [{\color{incolor}8}]:} \PY{n}{cancer}\PY{p}{[}\PY{l+s+s1}{\PYZsq{}}\PY{l+s+s1}{target}\PY{l+s+s1}{\PYZsq{}}\PY{p}{]}
\end{Verbatim}


\begin{Verbatim}[commandchars=\\\{\}]
{\color{outcolor}Out[{\color{outcolor}8}]:} array([0, 0, 0, 0, 0, 0, 0, 0, 0, 0, 0, 0, 0, 0, 0, 0, 0, 0, 0, 1, 1, 1,
               0, 0, 0, 0, 0, 0, 0, 0, 0, 0, 0, 0, 0, 0, 0, 1, 0, 0, 0, 0, 0, 0,
               0, 0, 1, 0, 1, 1, 1, 1, 1, 0, 0, 1, 0, 0, 1, 1, 1, 1, 0, 1, 0, 0,
               1, 1, 1, 1, 0, 1, 0, 0, 1, 0, 1, 0, 0, 1, 1, 1, 0, 0, 1, 0, 0, 0,
               1, 1, 1, 0, 1, 1, 0, 0, 1, 1, 1, 0, 0, 1, 1, 1, 1, 0, 1, 1, 0, 1,
               1, 1, 1, 1, 1, 1, 1, 0, 0, 0, 1, 0, 0, 1, 1, 1, 0, 0, 1, 0, 1, 0,
               0, 1, 0, 0, 1, 1, 0, 1, 1, 0, 1, 1, 1, 1, 0, 1, 1, 1, 1, 1, 1, 1,
               1, 1, 0, 1, 1, 1, 1, 0, 0, 1, 0, 1, 1, 0, 0, 1, 1, 0, 0, 1, 1, 1,
               1, 0, 1, 1, 0, 0, 0, 1, 0, 1, 0, 1, 1, 1, 0, 1, 1, 0, 0, 1, 0, 0,
               0, 0, 1, 0, 0, 0, 1, 0, 1, 0, 1, 1, 0, 1, 0, 0, 0, 0, 1, 1, 0, 0,
               1, 1, 1, 0, 1, 1, 1, 1, 1, 0, 0, 1, 1, 0, 1, 1, 0, 0, 1, 0, 1, 1,
               1, 1, 0, 1, 1, 1, 1, 1, 0, 1, 0, 0, 0, 0, 0, 0, 0, 0, 0, 0, 0, 0,
               0, 0, 1, 1, 1, 1, 1, 1, 0, 1, 0, 1, 1, 0, 1, 1, 0, 1, 0, 0, 1, 1,
               1, 1, 1, 1, 1, 1, 1, 1, 1, 1, 1, 0, 1, 1, 0, 1, 0, 1, 1, 1, 1, 1,
               1, 1, 1, 1, 1, 1, 1, 1, 1, 0, 1, 1, 1, 0, 1, 0, 1, 1, 1, 1, 0, 0,
               0, 1, 1, 1, 1, 0, 1, 0, 1, 0, 1, 1, 1, 0, 1, 1, 1, 1, 1, 1, 1, 0,
               0, 0, 1, 1, 1, 1, 1, 1, 1, 1, 1, 1, 1, 0, 0, 1, 0, 0, 0, 1, 0, 0,
               1, 1, 1, 1, 1, 0, 1, 1, 1, 1, 1, 0, 1, 1, 1, 0, 1, 1, 0, 0, 1, 1,
               1, 1, 1, 1, 0, 1, 1, 1, 1, 1, 1, 1, 0, 1, 1, 1, 1, 1, 0, 1, 1, 0,
               1, 1, 1, 1, 1, 1, 1, 1, 1, 1, 1, 1, 0, 1, 0, 0, 1, 0, 1, 1, 1, 1,
               1, 0, 1, 1, 0, 1, 0, 1, 1, 0, 1, 0, 1, 1, 1, 1, 1, 1, 1, 1, 0, 0,
               1, 1, 1, 1, 1, 1, 0, 1, 1, 1, 1, 1, 1, 1, 1, 1, 1, 0, 1, 1, 1, 1,
               1, 1, 1, 0, 1, 0, 1, 1, 0, 1, 1, 1, 1, 1, 0, 0, 1, 0, 1, 0, 1, 1,
               1, 1, 1, 0, 1, 1, 0, 1, 0, 1, 0, 0, 1, 1, 1, 0, 1, 1, 1, 1, 1, 1,
               1, 1, 1, 1, 1, 0, 1, 0, 0, 1, 1, 1, 1, 1, 1, 1, 1, 1, 1, 1, 1, 1,
               1, 1, 1, 1, 1, 1, 1, 1, 1, 1, 1, 1, 0, 0, 0, 0, 0, 0, 1])
\end{Verbatim}
            
    \begin{Verbatim}[commandchars=\\\{\}]
{\color{incolor}In [{\color{incolor}12}]:} \PY{n}{df\PYZus{}target} \PY{o}{=} \PY{n}{pd}\PY{o}{.}\PY{n}{DataFrame}\PY{p}{(}\PY{n}{cancer}\PY{p}{[}\PY{l+s+s1}{\PYZsq{}}\PY{l+s+s1}{target}\PY{l+s+s1}{\PYZsq{}}\PY{p}{]}\PY{p}{,}\PY{n}{columns}\PY{o}{=}\PY{p}{[}\PY{l+s+s1}{\PYZsq{}}\PY{l+s+s1}{Cancer}\PY{l+s+s1}{\PYZsq{}}\PY{p}{]}\PY{p}{)}
\end{Verbatim}


    Now let's actually check out the dataframe!

    \begin{Verbatim}[commandchars=\\\{\}]
{\color{incolor}In [{\color{incolor}9}]:} \PY{n}{df\PYZus{}feat}\PY{o}{.}\PY{n}{head}\PY{p}{(}\PY{p}{)}
\end{Verbatim}


\begin{Verbatim}[commandchars=\\\{\}]
{\color{outcolor}Out[{\color{outcolor}9}]:}    mean radius  mean texture  mean perimeter  mean area  mean smoothness  \textbackslash{}
        0        17.99         10.38          122.80     1001.0          0.11840   
        1        20.57         17.77          132.90     1326.0          0.08474   
        2        19.69         21.25          130.00     1203.0          0.10960   
        3        11.42         20.38           77.58      386.1          0.14250   
        4        20.29         14.34          135.10     1297.0          0.10030   
        
           mean compactness  mean concavity  mean concave points  mean symmetry  \textbackslash{}
        0           0.27760          0.3001              0.14710         0.2419   
        1           0.07864          0.0869              0.07017         0.1812   
        2           0.15990          0.1974              0.12790         0.2069   
        3           0.28390          0.2414              0.10520         0.2597   
        4           0.13280          0.1980              0.10430         0.1809   
        
           mean fractal dimension           {\ldots}             worst radius  \textbackslash{}
        0                 0.07871           {\ldots}                    25.38   
        1                 0.05667           {\ldots}                    24.99   
        2                 0.05999           {\ldots}                    23.57   
        3                 0.09744           {\ldots}                    14.91   
        4                 0.05883           {\ldots}                    22.54   
        
           worst texture  worst perimeter  worst area  worst smoothness  \textbackslash{}
        0          17.33           184.60      2019.0            0.1622   
        1          23.41           158.80      1956.0            0.1238   
        2          25.53           152.50      1709.0            0.1444   
        3          26.50            98.87       567.7            0.2098   
        4          16.67           152.20      1575.0            0.1374   
        
           worst compactness  worst concavity  worst concave points  worst symmetry  \textbackslash{}
        0             0.6656           0.7119                0.2654          0.4601   
        1             0.1866           0.2416                0.1860          0.2750   
        2             0.4245           0.4504                0.2430          0.3613   
        3             0.8663           0.6869                0.2575          0.6638   
        4             0.2050           0.4000                0.1625          0.2364   
        
           worst fractal dimension  
        0                  0.11890  
        1                  0.08902  
        2                  0.08758  
        3                  0.17300  
        4                  0.07678  
        
        [5 rows x 30 columns]
\end{Verbatim}
            
    \section{Exploratory Data Analysis}\label{exploratory-data-analysis}

    We'll skip the Data Viz part for this lecture since there are so many
features that are hard to interpret if you don't have domain knowledge
of cancer or tumor cells. In your project you will have more to
visualize for the data.

    \subsection{Train Test Split}\label{train-test-split}

    \begin{Verbatim}[commandchars=\\\{\}]
{\color{incolor}In [{\color{incolor}10}]:} \PY{k+kn}{from} \PY{n+nn}{sklearn}\PY{n+nn}{.}\PY{n+nn}{model\PYZus{}selection} \PY{k}{import} \PY{n}{train\PYZus{}test\PYZus{}split}
\end{Verbatim}


    \begin{Verbatim}[commandchars=\\\{\}]
{\color{incolor}In [{\color{incolor}13}]:} \PY{n}{X\PYZus{}train}\PY{p}{,} \PY{n}{X\PYZus{}test}\PY{p}{,} \PY{n}{y\PYZus{}train}\PY{p}{,} \PY{n}{y\PYZus{}test} \PY{o}{=} \PY{n}{train\PYZus{}test\PYZus{}split}\PY{p}{(}\PY{n}{df\PYZus{}feat}\PY{p}{,} \PY{n}{np}\PY{o}{.}\PY{n}{ravel}\PY{p}{(}\PY{n}{df\PYZus{}target}\PY{p}{)}\PY{p}{,} \PY{n}{test\PYZus{}size}\PY{o}{=}\PY{l+m+mf}{0.30}\PY{p}{,} \PY{n}{random\PYZus{}state}\PY{o}{=}\PY{l+m+mi}{101}\PY{p}{)}
\end{Verbatim}


    \section{Train the Support Vector
Classifier}\label{train-the-support-vector-classifier}

    \begin{Verbatim}[commandchars=\\\{\}]
{\color{incolor}In [{\color{incolor}14}]:} \PY{k+kn}{from} \PY{n+nn}{sklearn}\PY{n+nn}{.}\PY{n+nn}{svm} \PY{k}{import} \PY{n}{SVC}
\end{Verbatim}


    \begin{Verbatim}[commandchars=\\\{\}]
{\color{incolor}In [{\color{incolor}15}]:} \PY{n}{model} \PY{o}{=} \PY{n}{SVC}\PY{p}{(}\PY{p}{)}
\end{Verbatim}


    \begin{Verbatim}[commandchars=\\\{\}]
{\color{incolor}In [{\color{incolor}16}]:} \PY{n}{model}\PY{o}{.}\PY{n}{fit}\PY{p}{(}\PY{n}{X\PYZus{}train}\PY{p}{,}\PY{n}{y\PYZus{}train}\PY{p}{)}
\end{Verbatim}


    \begin{Verbatim}[commandchars=\\\{\}]
C:\textbackslash{}Users\textbackslash{}admin\textbackslash{}Anaconda3\textbackslash{}lib\textbackslash{}site-packages\textbackslash{}sklearn\textbackslash{}svm\textbackslash{}base.py:196: FutureWarning: The default value of gamma will change from 'auto' to 'scale' in version 0.22 to account better for unscaled features. Set gamma explicitly to 'auto' or 'scale' to avoid this warning.
  "avoid this warning.", FutureWarning)

    \end{Verbatim}

\begin{Verbatim}[commandchars=\\\{\}]
{\color{outcolor}Out[{\color{outcolor}16}]:} SVC(C=1.0, cache\_size=200, class\_weight=None, coef0=0.0,
           decision\_function\_shape='ovr', degree=3, gamma='auto\_deprecated',
           kernel='rbf', max\_iter=-1, probability=False, random\_state=None,
           shrinking=True, tol=0.001, verbose=False)
\end{Verbatim}
            
    \subsection{Predictions and
Evaluations}\label{predictions-and-evaluations}

Now let's predict using the trained model.

    \begin{Verbatim}[commandchars=\\\{\}]
{\color{incolor}In [{\color{incolor}17}]:} \PY{n}{predictions} \PY{o}{=} \PY{n}{model}\PY{o}{.}\PY{n}{predict}\PY{p}{(}\PY{n}{X\PYZus{}test}\PY{p}{)}
\end{Verbatim}


    \begin{Verbatim}[commandchars=\\\{\}]
{\color{incolor}In [{\color{incolor}18}]:} \PY{k+kn}{from} \PY{n+nn}{sklearn}\PY{n+nn}{.}\PY{n+nn}{metrics} \PY{k}{import} \PY{n}{classification\PYZus{}report}\PY{p}{,}\PY{n}{confusion\PYZus{}matrix}
\end{Verbatim}


    \begin{Verbatim}[commandchars=\\\{\}]
{\color{incolor}In [{\color{incolor}19}]:} \PY{n+nb}{print}\PY{p}{(}\PY{n}{confusion\PYZus{}matrix}\PY{p}{(}\PY{n}{y\PYZus{}test}\PY{p}{,}\PY{n}{predictions}\PY{p}{)}\PY{p}{)}
\end{Verbatim}


    \begin{Verbatim}[commandchars=\\\{\}]
[[  0  66]
 [  0 105]]

    \end{Verbatim}

    \begin{Verbatim}[commandchars=\\\{\}]
{\color{incolor}In [{\color{incolor}20}]:} \PY{n+nb}{print}\PY{p}{(}\PY{n}{classification\PYZus{}report}\PY{p}{(}\PY{n}{y\PYZus{}test}\PY{p}{,}\PY{n}{predictions}\PY{p}{)}\PY{p}{)}
\end{Verbatim}


    \begin{Verbatim}[commandchars=\\\{\}]
              precision    recall  f1-score   support

           0       0.00      0.00      0.00        66
           1       0.61      1.00      0.76       105

   micro avg       0.61      0.61      0.61       171
   macro avg       0.31      0.50      0.38       171
weighted avg       0.38      0.61      0.47       171


    \end{Verbatim}

    \begin{Verbatim}[commandchars=\\\{\}]
C:\textbackslash{}Users\textbackslash{}admin\textbackslash{}Anaconda3\textbackslash{}lib\textbackslash{}site-packages\textbackslash{}sklearn\textbackslash{}metrics\textbackslash{}classification.py:1143: UndefinedMetricWarning: Precision and F-score are ill-defined and being set to 0.0 in labels with no predicted samples.
  'precision', 'predicted', average, warn\_for)
C:\textbackslash{}Users\textbackslash{}admin\textbackslash{}Anaconda3\textbackslash{}lib\textbackslash{}site-packages\textbackslash{}sklearn\textbackslash{}metrics\textbackslash{}classification.py:1143: UndefinedMetricWarning: Precision and F-score are ill-defined and being set to 0.0 in labels with no predicted samples.
  'precision', 'predicted', average, warn\_for)
C:\textbackslash{}Users\textbackslash{}admin\textbackslash{}Anaconda3\textbackslash{}lib\textbackslash{}site-packages\textbackslash{}sklearn\textbackslash{}metrics\textbackslash{}classification.py:1143: UndefinedMetricWarning: Precision and F-score are ill-defined and being set to 0.0 in labels with no predicted samples.
  'precision', 'predicted', average, warn\_for)

    \end{Verbatim}

    Woah! Notice that we are classifying everything into a single class!
This means our model needs to have it parameters adjusted (it may also
help to normalize the data).

We can search for parameters using a GridSearch!

    \section{Gridsearch}\label{gridsearch}

Finding the right parameters (like what C or gamma values to use) is a
tricky task! But luckily, we can be a little lazy and just try a bunch
of combinations and see what works best! This idea of creating a 'grid'
of parameters and just trying out all the possible combinations is
called a Gridsearch, this method is common enough that Scikit-learn has
this functionality built in with GridSearchCV! The CV stands for
cross-validation which is the

GridSearchCV takes a dictionary that describes the parameters that
should be tried and a model to train. The grid of parameters is defined
as a dictionary, where the keys are the parameters and the values are
the settings to be tested.

    \begin{Verbatim}[commandchars=\\\{\}]
{\color{incolor}In [{\color{incolor}21}]:} \PY{n}{param\PYZus{}grid} \PY{o}{=} \PY{p}{\PYZob{}}\PY{l+s+s1}{\PYZsq{}}\PY{l+s+s1}{C}\PY{l+s+s1}{\PYZsq{}}\PY{p}{:} \PY{p}{[}\PY{l+m+mf}{0.1}\PY{p}{,}\PY{l+m+mi}{1}\PY{p}{,} \PY{l+m+mi}{10}\PY{p}{,} \PY{l+m+mi}{100}\PY{p}{,} \PY{l+m+mi}{1000}\PY{p}{]}\PY{p}{,} \PY{l+s+s1}{\PYZsq{}}\PY{l+s+s1}{gamma}\PY{l+s+s1}{\PYZsq{}}\PY{p}{:} \PY{p}{[}\PY{l+m+mi}{1}\PY{p}{,}\PY{l+m+mf}{0.1}\PY{p}{,}\PY{l+m+mf}{0.01}\PY{p}{,}\PY{l+m+mf}{0.001}\PY{p}{,}\PY{l+m+mf}{0.0001}\PY{p}{]}\PY{p}{,} \PY{l+s+s1}{\PYZsq{}}\PY{l+s+s1}{kernel}\PY{l+s+s1}{\PYZsq{}}\PY{p}{:} \PY{p}{[}\PY{l+s+s1}{\PYZsq{}}\PY{l+s+s1}{rbf}\PY{l+s+s1}{\PYZsq{}}\PY{p}{]}\PY{p}{\PYZcb{}} 
\end{Verbatim}


    \begin{Verbatim}[commandchars=\\\{\}]
{\color{incolor}In [{\color{incolor}22}]:} \PY{k+kn}{from} \PY{n+nn}{sklearn}\PY{n+nn}{.}\PY{n+nn}{model\PYZus{}selection} \PY{k}{import} \PY{n}{GridSearchCV}
\end{Verbatim}


    One of the great things about GridSearchCV is that it is a
meta-estimator. It takes an estimator like SVC, and creates a new
estimator, that behaves exactly the same - in this case, like a
classifier. You should add refit=True and choose verbose to whatever
number you want, higher the number, the more verbose (verbose just means
the text output describing the process).

    \begin{Verbatim}[commandchars=\\\{\}]
{\color{incolor}In [{\color{incolor}23}]:} \PY{n}{grid} \PY{o}{=} \PY{n}{GridSearchCV}\PY{p}{(}\PY{n}{SVC}\PY{p}{(}\PY{p}{)}\PY{p}{,}\PY{n}{param\PYZus{}grid}\PY{p}{,}\PY{n}{refit}\PY{o}{=}\PY{k+kc}{True}\PY{p}{,}\PY{n}{verbose}\PY{o}{=}\PY{l+m+mi}{3}\PY{p}{)}
\end{Verbatim}


    What fit does is a bit more involved then usual. First, it runs the same
loop with cross-validation, to find the best parameter combination. Once
it has the best combination, it runs fit again on all data passed to fit
(without cross-validation), to built a single new model using the best
parameter setting.

    \begin{Verbatim}[commandchars=\\\{\}]
{\color{incolor}In [{\color{incolor}24}]:} \PY{c+c1}{\PYZsh{} May take awhile!}
         \PY{n}{grid}\PY{o}{.}\PY{n}{fit}\PY{p}{(}\PY{n}{X\PYZus{}train}\PY{p}{,}\PY{n}{y\PYZus{}train}\PY{p}{)}
\end{Verbatim}


    \begin{Verbatim}[commandchars=\\\{\}]
C:\textbackslash{}Users\textbackslash{}admin\textbackslash{}Anaconda3\textbackslash{}lib\textbackslash{}site-packages\textbackslash{}sklearn\textbackslash{}model\_selection\textbackslash{}\_split.py:2053: FutureWarning: You should specify a value for 'cv' instead of relying on the default value. The default value will change from 3 to 5 in version 0.22.
  warnings.warn(CV\_WARNING, FutureWarning)
[Parallel(n\_jobs=1)]: Using backend SequentialBackend with 1 concurrent workers.
[Parallel(n\_jobs=1)]: Done   1 out of   1 | elapsed:    0.0s remaining:    0.0s
[Parallel(n\_jobs=1)]: Done   2 out of   2 | elapsed:    0.0s remaining:    0.0s

    \end{Verbatim}

    \begin{Verbatim}[commandchars=\\\{\}]
Fitting 3 folds for each of 25 candidates, totalling 75 fits
[CV] C=0.1, gamma=1, kernel=rbf {\ldots}
[CV]  C=0.1, gamma=1, kernel=rbf, score=0.631578947368421, total=   0.0s
[CV] C=0.1, gamma=1, kernel=rbf {\ldots}
[CV]  C=0.1, gamma=1, kernel=rbf, score=0.631578947368421, total=   0.0s
[CV] C=0.1, gamma=1, kernel=rbf {\ldots}
[CV]  C=0.1, gamma=1, kernel=rbf, score=0.6363636363636364, total=   0.0s
[CV] C=0.1, gamma=0.1, kernel=rbf {\ldots}
[CV]  C=0.1, gamma=0.1, kernel=rbf, score=0.631578947368421, total=   0.0s
[CV] C=0.1, gamma=0.1, kernel=rbf {\ldots}
[CV]  C=0.1, gamma=0.1, kernel=rbf, score=0.631578947368421, total=   0.0s
[CV] C=0.1, gamma=0.1, kernel=rbf {\ldots}
[CV]  C=0.1, gamma=0.1, kernel=rbf, score=0.6363636363636364, total=   0.0s
[CV] C=0.1, gamma=0.01, kernel=rbf {\ldots}
[CV]  C=0.1, gamma=0.01, kernel=rbf, score=0.631578947368421, total=   0.0s
[CV] C=0.1, gamma=0.01, kernel=rbf {\ldots}
[CV]  C=0.1, gamma=0.01, kernel=rbf, score=0.631578947368421, total=   0.0s
[CV] C=0.1, gamma=0.01, kernel=rbf {\ldots}
[CV]  C=0.1, gamma=0.01, kernel=rbf, score=0.6363636363636364, total=   0.0s
[CV] C=0.1, gamma=0.001, kernel=rbf {\ldots}
[CV]  C=0.1, gamma=0.001, kernel=rbf, score=0.631578947368421, total=   0.0s
[CV] C=0.1, gamma=0.001, kernel=rbf {\ldots}
[CV]  C=0.1, gamma=0.001, kernel=rbf, score=0.631578947368421, total=   0.0s
[CV] C=0.1, gamma=0.001, kernel=rbf {\ldots}
[CV]  C=0.1, gamma=0.001, kernel=rbf, score=0.6363636363636364, total=   0.0s
[CV] C=0.1, gamma=0.0001, kernel=rbf {\ldots}
[CV]  C=0.1, gamma=0.0001, kernel=rbf, score=0.9022556390977443, total=   0.0s
[CV] C=0.1, gamma=0.0001, kernel=rbf {\ldots}
[CV]  C=0.1, gamma=0.0001, kernel=rbf, score=0.9624060150375939, total=   0.0s
[CV] C=0.1, gamma=0.0001, kernel=rbf {\ldots}
[CV]  C=0.1, gamma=0.0001, kernel=rbf, score=0.9166666666666666, total=   0.0s
[CV] C=1, gamma=1, kernel=rbf {\ldots}
[CV]  C=1, gamma=1, kernel=rbf, score=0.631578947368421, total=   0.0s
[CV] C=1, gamma=1, kernel=rbf {\ldots}
[CV]  C=1, gamma=1, kernel=rbf, score=0.631578947368421, total=   0.0s
[CV] C=1, gamma=1, kernel=rbf {\ldots}
[CV]  C=1, gamma=1, kernel=rbf, score=0.6363636363636364, total=   0.0s
[CV] C=1, gamma=0.1, kernel=rbf {\ldots}
[CV]  C=1, gamma=0.1, kernel=rbf, score=0.631578947368421, total=   0.0s
[CV] C=1, gamma=0.1, kernel=rbf {\ldots}
[CV]  C=1, gamma=0.1, kernel=rbf, score=0.631578947368421, total=   0.0s
[CV] C=1, gamma=0.1, kernel=rbf {\ldots}
[CV]  C=1, gamma=0.1, kernel=rbf, score=0.6363636363636364, total=   0.0s
[CV] C=1, gamma=0.01, kernel=rbf {\ldots}
[CV]  C=1, gamma=0.01, kernel=rbf, score=0.631578947368421, total=   0.0s
[CV] C=1, gamma=0.01, kernel=rbf {\ldots}
[CV]  C=1, gamma=0.01, kernel=rbf, score=0.631578947368421, total=   0.0s
[CV] C=1, gamma=0.01, kernel=rbf {\ldots}
[CV]  C=1, gamma=0.01, kernel=rbf, score=0.6363636363636364, total=   0.0s
[CV] C=1, gamma=0.001, kernel=rbf {\ldots}
[CV]  C=1, gamma=0.001, kernel=rbf, score=0.9022556390977443, total=   0.0s
[CV] C=1, gamma=0.001, kernel=rbf {\ldots}
[CV]  C=1, gamma=0.001, kernel=rbf, score=0.9398496240601504, total=   0.0s
[CV] C=1, gamma=0.001, kernel=rbf {\ldots}
[CV]  C=1, gamma=0.001, kernel=rbf, score=0.9545454545454546, total=   0.0s
[CV] C=1, gamma=0.0001, kernel=rbf {\ldots}
[CV]  C=1, gamma=0.0001, kernel=rbf, score=0.9398496240601504, total=   0.0s
[CV] C=1, gamma=0.0001, kernel=rbf {\ldots}
[CV]  C=1, gamma=0.0001, kernel=rbf, score=0.9699248120300752, total=   0.0s
[CV] C=1, gamma=0.0001, kernel=rbf {\ldots}
[CV]  C=1, gamma=0.0001, kernel=rbf, score=0.946969696969697, total=   0.0s
[CV] C=10, gamma=1, kernel=rbf {\ldots}
[CV]  C=10, gamma=1, kernel=rbf, score=0.631578947368421, total=   0.0s
[CV] C=10, gamma=1, kernel=rbf {\ldots}
[CV]  C=10, gamma=1, kernel=rbf, score=0.631578947368421, total=   0.0s
[CV] C=10, gamma=1, kernel=rbf {\ldots}
[CV]  C=10, gamma=1, kernel=rbf, score=0.6363636363636364, total=   0.0s
[CV] C=10, gamma=0.1, kernel=rbf {\ldots}
[CV]  C=10, gamma=0.1, kernel=rbf, score=0.631578947368421, total=   0.0s
[CV] C=10, gamma=0.1, kernel=rbf {\ldots}
[CV]  C=10, gamma=0.1, kernel=rbf, score=0.631578947368421, total=   0.0s
[CV] C=10, gamma=0.1, kernel=rbf {\ldots}
[CV]  C=10, gamma=0.1, kernel=rbf, score=0.6363636363636364, total=   0.0s
[CV] C=10, gamma=0.01, kernel=rbf {\ldots}
[CV]  C=10, gamma=0.01, kernel=rbf, score=0.631578947368421, total=   0.0s
[CV] C=10, gamma=0.01, kernel=rbf {\ldots}
[CV]  C=10, gamma=0.01, kernel=rbf, score=0.631578947368421, total=   0.0s
[CV] C=10, gamma=0.01, kernel=rbf {\ldots}
[CV]  C=10, gamma=0.01, kernel=rbf, score=0.6363636363636364, total=   0.0s
[CV] C=10, gamma=0.001, kernel=rbf {\ldots}
[CV]  C=10, gamma=0.001, kernel=rbf, score=0.8947368421052632, total=   0.0s
[CV] C=10, gamma=0.001, kernel=rbf {\ldots}
[CV]  C=10, gamma=0.001, kernel=rbf, score=0.9323308270676691, total=   0.0s
[CV] C=10, gamma=0.001, kernel=rbf {\ldots}
[CV]  C=10, gamma=0.001, kernel=rbf, score=0.9166666666666666, total=   0.0s
[CV] C=10, gamma=0.0001, kernel=rbf {\ldots}
[CV]  C=10, gamma=0.0001, kernel=rbf, score=0.9323308270676691, total=   0.0s
[CV] C=10, gamma=0.0001, kernel=rbf {\ldots}
[CV]  C=10, gamma=0.0001, kernel=rbf, score=0.9699248120300752, total=   0.0s
[CV] C=10, gamma=0.0001, kernel=rbf {\ldots}
[CV]  C=10, gamma=0.0001, kernel=rbf, score=0.9621212121212122, total=   0.0s
[CV] C=100, gamma=1, kernel=rbf {\ldots}
[CV]  C=100, gamma=1, kernel=rbf, score=0.631578947368421, total=   0.0s
[CV] C=100, gamma=1, kernel=rbf {\ldots}
[CV]  C=100, gamma=1, kernel=rbf, score=0.631578947368421, total=   0.0s
[CV] C=100, gamma=1, kernel=rbf {\ldots}
[CV]  C=100, gamma=1, kernel=rbf, score=0.6363636363636364, total=   0.0s
[CV] C=100, gamma=0.1, kernel=rbf {\ldots}
[CV]  C=100, gamma=0.1, kernel=rbf, score=0.631578947368421, total=   0.0s
[CV] C=100, gamma=0.1, kernel=rbf {\ldots}
[CV]  C=100, gamma=0.1, kernel=rbf, score=0.631578947368421, total=   0.0s
[CV] C=100, gamma=0.1, kernel=rbf {\ldots}
[CV]  C=100, gamma=0.1, kernel=rbf, score=0.6363636363636364, total=   0.0s
[CV] C=100, gamma=0.01, kernel=rbf {\ldots}
[CV]  C=100, gamma=0.01, kernel=rbf, score=0.631578947368421, total=   0.0s
[CV] C=100, gamma=0.01, kernel=rbf {\ldots}
[CV]  C=100, gamma=0.01, kernel=rbf, score=0.631578947368421, total=   0.0s
[CV] C=100, gamma=0.01, kernel=rbf {\ldots}
[CV]  C=100, gamma=0.01, kernel=rbf, score=0.6363636363636364, total=   0.0s
[CV] C=100, gamma=0.001, kernel=rbf {\ldots}
[CV]  C=100, gamma=0.001, kernel=rbf, score=0.8947368421052632, total=   0.0s
[CV] C=100, gamma=0.001, kernel=rbf {\ldots}
[CV]  C=100, gamma=0.001, kernel=rbf, score=0.9323308270676691, total=   0.0s
[CV] C=100, gamma=0.001, kernel=rbf {\ldots}
[CV]  C=100, gamma=0.001, kernel=rbf, score=0.9166666666666666, total=   0.0s
[CV] C=100, gamma=0.0001, kernel=rbf {\ldots}
[CV]  C=100, gamma=0.0001, kernel=rbf, score=0.9172932330827067, total=   0.0s
[CV] C=100, gamma=0.0001, kernel=rbf {\ldots}
[CV]  C=100, gamma=0.0001, kernel=rbf, score=0.9774436090225563, total=   0.0s
[CV] C=100, gamma=0.0001, kernel=rbf {\ldots}
[CV]  C=100, gamma=0.0001, kernel=rbf, score=0.9393939393939394, total=   0.0s
[CV] C=1000, gamma=1, kernel=rbf {\ldots}
[CV]  C=1000, gamma=1, kernel=rbf, score=0.631578947368421, total=   0.0s
[CV] C=1000, gamma=1, kernel=rbf {\ldots}
[CV]  C=1000, gamma=1, kernel=rbf, score=0.631578947368421, total=   0.0s
[CV] C=1000, gamma=1, kernel=rbf {\ldots}
[CV]  C=1000, gamma=1, kernel=rbf, score=0.6363636363636364, total=   0.0s
[CV] C=1000, gamma=0.1, kernel=rbf {\ldots}
[CV]  C=1000, gamma=0.1, kernel=rbf, score=0.631578947368421, total=   0.0s
[CV] C=1000, gamma=0.1, kernel=rbf {\ldots}
[CV]  C=1000, gamma=0.1, kernel=rbf, score=0.631578947368421, total=   0.0s
[CV] C=1000, gamma=0.1, kernel=rbf {\ldots}
[CV]  C=1000, gamma=0.1, kernel=rbf, score=0.6363636363636364, total=   0.0s
[CV] C=1000, gamma=0.01, kernel=rbf {\ldots}
[CV]  C=1000, gamma=0.01, kernel=rbf, score=0.631578947368421, total=   0.0s
[CV] C=1000, gamma=0.01, kernel=rbf {\ldots}
[CV]  C=1000, gamma=0.01, kernel=rbf, score=0.631578947368421, total=   0.0s
[CV] C=1000, gamma=0.01, kernel=rbf {\ldots}
[CV]  C=1000, gamma=0.01, kernel=rbf, score=0.6363636363636364, total=   0.0s
[CV] C=1000, gamma=0.001, kernel=rbf {\ldots}
[CV]  C=1000, gamma=0.001, kernel=rbf, score=0.8947368421052632, total=   0.0s
[CV] C=1000, gamma=0.001, kernel=rbf {\ldots}
[CV]  C=1000, gamma=0.001, kernel=rbf, score=0.9323308270676691, total=   0.0s
[CV] C=1000, gamma=0.001, kernel=rbf {\ldots}
[CV]  C=1000, gamma=0.001, kernel=rbf, score=0.9166666666666666, total=   0.0s
[CV] C=1000, gamma=0.0001, kernel=rbf {\ldots}
[CV]  C=1000, gamma=0.0001, kernel=rbf, score=0.9097744360902256, total=   0.0s
[CV] C=1000, gamma=0.0001, kernel=rbf {\ldots}
[CV]  C=1000, gamma=0.0001, kernel=rbf, score=0.9699248120300752, total=   0.0s
[CV] C=1000, gamma=0.0001, kernel=rbf {\ldots}
[CV]  C=1000, gamma=0.0001, kernel=rbf, score=0.9318181818181818, total=   0.0s

    \end{Verbatim}

    \begin{Verbatim}[commandchars=\\\{\}]
[Parallel(n\_jobs=1)]: Done  75 out of  75 | elapsed:    1.1s finished

    \end{Verbatim}

\begin{Verbatim}[commandchars=\\\{\}]
{\color{outcolor}Out[{\color{outcolor}24}]:} GridSearchCV(cv='warn', error\_score='raise-deprecating',
                estimator=SVC(C=1.0, cache\_size=200, class\_weight=None, coef0=0.0,
           decision\_function\_shape='ovr', degree=3, gamma='auto\_deprecated',
           kernel='rbf', max\_iter=-1, probability=False, random\_state=None,
           shrinking=True, tol=0.001, verbose=False),
                fit\_params=None, iid='warn', n\_jobs=None,
                param\_grid=\{'C': [0.1, 1, 10, 100, 1000], 'gamma': [1, 0.1, 0.01, 0.001, 0.0001], 'kernel': ['rbf']\},
                pre\_dispatch='2*n\_jobs', refit=True, return\_train\_score='warn',
                scoring=None, verbose=3)
\end{Verbatim}
            
    You can inspect the best parameters found by GridSearchCV in the
best\_params\_ attribute, and the best estimator in the
best\_estimator\_ attribute:

    \begin{Verbatim}[commandchars=\\\{\}]
{\color{incolor}In [{\color{incolor}25}]:} \PY{n}{grid}\PY{o}{.}\PY{n}{best\PYZus{}params\PYZus{}}
\end{Verbatim}


\begin{Verbatim}[commandchars=\\\{\}]
{\color{outcolor}Out[{\color{outcolor}25}]:} \{'C': 10, 'gamma': 0.0001, 'kernel': 'rbf'\}
\end{Verbatim}
            
    \begin{Verbatim}[commandchars=\\\{\}]
{\color{incolor}In [{\color{incolor}26}]:} \PY{n}{grid}\PY{o}{.}\PY{n}{best\PYZus{}estimator\PYZus{}}
\end{Verbatim}


\begin{Verbatim}[commandchars=\\\{\}]
{\color{outcolor}Out[{\color{outcolor}26}]:} SVC(C=10, cache\_size=200, class\_weight=None, coef0=0.0,
           decision\_function\_shape='ovr', degree=3, gamma=0.0001, kernel='rbf',
           max\_iter=-1, probability=False, random\_state=None, shrinking=True,
           tol=0.001, verbose=False)
\end{Verbatim}
            
    Then you can re-run predictions on this grid object just like you would
with a normal model.

    \begin{Verbatim}[commandchars=\\\{\}]
{\color{incolor}In [{\color{incolor}27}]:} \PY{n}{grid\PYZus{}predictions} \PY{o}{=} \PY{n}{grid}\PY{o}{.}\PY{n}{predict}\PY{p}{(}\PY{n}{X\PYZus{}test}\PY{p}{)}
\end{Verbatim}


    \begin{Verbatim}[commandchars=\\\{\}]
{\color{incolor}In [{\color{incolor}28}]:} \PY{n+nb}{print}\PY{p}{(}\PY{n}{confusion\PYZus{}matrix}\PY{p}{(}\PY{n}{y\PYZus{}test}\PY{p}{,}\PY{n}{grid\PYZus{}predictions}\PY{p}{)}\PY{p}{)}
\end{Verbatim}


    \begin{Verbatim}[commandchars=\\\{\}]
[[ 60   6]
 [  3 102]]

    \end{Verbatim}

    \begin{Verbatim}[commandchars=\\\{\}]
{\color{incolor}In [{\color{incolor}29}]:} \PY{n+nb}{print}\PY{p}{(}\PY{n}{classification\PYZus{}report}\PY{p}{(}\PY{n}{y\PYZus{}test}\PY{p}{,}\PY{n}{grid\PYZus{}predictions}\PY{p}{)}\PY{p}{)}
\end{Verbatim}


    \begin{Verbatim}[commandchars=\\\{\}]
              precision    recall  f1-score   support

           0       0.95      0.91      0.93        66
           1       0.94      0.97      0.96       105

   micro avg       0.95      0.95      0.95       171
   macro avg       0.95      0.94      0.94       171
weighted avg       0.95      0.95      0.95       171


    \end{Verbatim}


    % Add a bibliography block to the postdoc
    
    
    
    \end{document}
